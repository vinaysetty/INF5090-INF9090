% THIS IS SIGPROC-SP.TEX - VERSION 3.1
% WORKS WITH V3.2SP OF ACM_PROC_ARTICLE-SP.CLS
% APRIL 2009
%
% ----------------------------------------------------------------------------------------------------------------
% This .tex file (and associated .cls V3.2SP) *DOES NOT* produce:
%       1) The Permission Statement
%       2) The Conference (location) Info information
%       3) The Copyright Line with ACM data
%       4) Page numbering
% ---------------------------------------------------------------------------------------------------------------
% It is an example which *does* use the .bib file (from which the .bbl file
% is produced).
% REMEMBER HOWEVER: After having produced the .bbl file,
% and prior to final submission,
% you need to 'insert'  your .bbl file into your source .tex file so as to provide
% ONE 'self-contained' source file.
%

\documentclass[12pt, conference, compsocconf, letterpaper]{IEEEtran}
%\documentclass{sig-alternate}

%\usepackage{amssymb}
\usepackage{algorithmic,algorithm, graphicx}
\usepackage{paralist}
\usepackage{url}
\usepackage{authblk}
  
\newcommand{\dims}{\mathsf{d}}

%\DeclareMathOperator*{\argmin}{arg\,min}

\begin{document}

\title{INF5090/9090\\State Exchange in Distributed Applications}

%\subtitle{[Extended Abstract]
%\titlenote{A full version of this paper is available as
%\textit{Author's Guide to Preparing ACM SIG Proceedings Using
%\LaTeX$2_\epsilon$\ and BibTeX} at
%\texttt{www.acm.org/eaddress.htm}}}
%
% You need the command \numberofauthors to handle the 'placement
% and alignment' of the authors beneath the title.
%
% For aesthetic reasons, we recommend 'three authors at a time'
% i.e. three 'name/affiliation blocks' be placed beneath the title.
%
% NOTE: You are NOT restricted in how many 'rows' of
% "name/affiliations" may appear. We just ask that you restrict
% the number of 'columns' to three.
%
% Because of the available 'opening page real-estate'
% we ask you to refrain from putting more than six authors
% (two rows with three columns) beneath the article title.
% More than six makes the first-page appear very cluttered indeed.
%
% Use the \alignauthor commands to handle the names
% and affiliations for an 'aesthetic maximum' of six authors.
% Add names, affiliations, addresses for
% the seventh etc. author(s) as the argument for the
% \additionalauthors command.
% These 'additional authors' will be output/set for you
% without further effort on your part as the last section in
% the body of your article BEFORE References or any Appendices.

%\numberofauthors{3} %  in this sample file, there are a *total*
% of EIGHT authors. SIX appear on the 'first-page' (for formatting
% reasons) and the remaining two appear in the \additionalauthors section.
%
\author{
	Magnus Evensberget\\
	Institute of Informatics\\
	University of Oslo\\
	magnusev@ifi.uio.no
\and
	Vinay Setty\\
	Institute of Informatics\\
	University of Oslo\\
	vinay@ifi.uio.no
}

\maketitle
\begin{abstract}
In this assignment we analyze and evaluate the Real-time Application Mobility Platform (TRAMP) project with focus on distributed multimedia applications with real-time requirements. 
We design and implement two components (producer and consumer) of a real-time application (by simulating video streaming) and evaluate the provided distribution framework with respect to delay and packet loss. We analyze TRAMP and point out the weaknesses with the help of experiments. 
\end{abstract}

% A category with the (minimum) three required fields
%\category{H.4}{Information Systems Applications}{Miscellaneous}
%A category including the fourth, optional field follows...
%\category{D.2.8}{Software Engineering}{Metrics}[complexity measures, performance measures]

%\terms{Design, Experimentation, Performance, Measurement}



\section{Introduction}
In our assignment we were to use the Real-time Application Mobility Platform (TRAMP). This is a project developed by the Distributed Multimedia Systems (DMMS) group at the University of Oslo.

This is a assignment given to us in the course INF5090 - Advanced Topics in Distributed Systems which is a course given by the University of Oslo, as well as Lancaster University in England and University of Mannheim in Germany. The teachers are Thomas Plagemann and Vera Goebel with their teaching assistants Piotr Kaminski and Hans Vatne Hansen. These are also the people that developed the TRAMP framework together with some other developers all from the University of Oslo.


\section{Related work}
\label{sec:relatedwork}

\subsection{Skype}
As we've mentioned earlier in the report Skype is one of the applications the developers of the TRAMP framework reference to. Skype was the first peer-to-peer voice over IP (VoIP) network, and requires minimal infrastructure in order to function. Skype came on market in late 2003 developed by Niklas Zennstr�m and Janus Friis, the two main developer behind the filesharing system Kazaa. this was bought by eBay in late 2005 and again by Microsoft in May 2011.

The skype arcitecture is designed around three entities: Supernodes, regular nodes and the login server (See fig. 1). Each regular node has a cache of the IP address and portnumber to all reachable Supernodes. The Supernodes are a subset of the regular nodes. If a node has high uptime, good bandwidth and is not restricted by firewalls or Network Address translation (NAT) it can be chosen as a Supernode\cite{Analysis of Skype p2p}. This will ofcourse put some extra stress on the nodes not behind NAT, but also make Skype possible to work for free due to the little infrastructure needed in order to make the network work. Another issue for these Supernodes is that they are used as third party for UDP hole punching to connect clients behind NAT. UDP hole punching is done by letting the regular nodes behind NAT connect to the third party (the Supernode) thereby opening ports that they can use for direct traffic between the two regular nodes. This is only open as long as there is communication traffic going. If there is a prolonged absence of traffic Skype sends ``keep alive'' packets instead of closing the connection and then having to use the Supernode as a third party again redo the connection.\cite{UDP_holepunching}

This is something that might be implemented in TRAMP to scale up the network. 


\begin{figure}[hb]
%\centering
\includegraphics[width=200pt]{Skype_Architecture.png}
\caption{Picture of the skype arcitecture \cite{Analysis of Skype p2p}}
\end{figure}

\subsection{Network-Integrated Multimedia Middleware (NMM)}
http://www.networkmultimedia.org

%\begin{center}
 %\includegraphics[width=200pt]{Skype_Architecture.png}
 % Skype_Architecture.png: 347x532 pixel, 96dpi, 9.18x14.08 cm, bb=0 0 260 399
%\caption{Picture of the skype arcitecture} %\cite{Analysis of Skype p2p}}
%\end{center}

%\begin{center}
%\begin{figure*}
% \includegraphics[width=7.0in, height=3.5in]{img1.pdf}
 % img1.pdf: 0x0 pixel, 300dpi, 0.00x0.00 cm, bb=
%\end{figure*}
%\end{center}



%talk about a streaming application (pref p2p udp and tcp and what is trades / tradeoffs when using the different


\section{System Design}
\label{sec:design}


\section{The choice of streamer}
\label{sec:streamer}


\section{Proposed Optimizations}
\label{sec:optimizations}

\subsection{Shared Memory problem}

\subsection{Pushing more data than the recieving nodes can pull}

\subsection{UDP vs TCP or both}




%shared mem probelm
%sending data to fast (recieving too slow)

%delay(?)

%udp vs tcp or both

\section{Evaluation}
\label{sec:eval}

Our implementation is a file sharer that uses the TRAMP platform to distribute files \ref{sec:streamer}

\section{Conclusions}
In this assignment we were able to use TRAMP framework to simulate a real time application. We did thorough analysis of TRAMP framework and made several interesting observations. Mainly, we observed that TRAMP starts with a mesh network topology which is not scalable as the number of peers in the network grow. The subscribers establish a routing tree by choosing the peer with lowest network delay as neighbors. This approach is flawed due to two reasons: (a) the tree may not give least delay path from producer to consumers (b) there is a possibility that the tree has disconnections. We observe this behavior by measuring the end-to-end delay from producer to the consumers. We also observed that since tramp daemon writes the packets to the shared memory and since there is no queuing mechanism at the consumer side it could result in a significant packet loss due to overwriting of packets in shared memory. We also concluded that this can be avoided by introducing around 500ms delay between sending of packets.
%\section{Acknowledgments}



\bibliographystyle{abbrv}
\bibliography{main}  


%\balancecolumns

\end{document}
